\subsection{Решение за $O(n^3 + n^2S)$ времени}

Пусть вес запроса - это то же самое, что и его вероятность.

Пусть вес дерева отрезков - это сумма по всем данным запросам произведения их веса на количество посещенных вершин в дереве отрезков.

Введём понятие дерева отрезков, построенного на $[L, R]$ и его веса.
Корень такого дерева отрезков соответствует отрезку $[L, R]$. Запросы, не пересекающиеся с $[L, R]$ убираются из рассмотрения, а для остальных берется их пересечение с отрезком $[L, R]$.

Дальше несколько фактов:

\begin{itemize}
    \item Для отрезка $[L, R]$ к весу дерева отрезков прибавляется сумма весов запросов, которые пересекаются с $[L, R]$
    \item Если сыновьям корня соответствуют отрезки $[L, m]$ и $[m + 1, R]$, то в весе дерева отрезков на $[L, R]$ надо
    учесть вес дерева отрезков на $[L, m]$ и на $[m + 1, R]$.
    \item При этом запросы целиком содержащие $[L, R]$ не посещают никаких вершин кроме корня, поэтому их нужно
    вычесть из весов деревьев отрезков на $[L, m]$ и на $[m + 1, R]$.
\end{itemize}
    
Введем несколько обозначений:

\begin{itemize}
    \item $include_{L, R}$ - суммарный вес запросов, которые содержат отрезок $[L, R]$
    \item $intersect_{L, R}$ - суммарный вес запросов, которые пересекаются с отрезком $[L, R]$
    \item $dp_{L, R}$ - минимальный вес дерева отрезков, построенного на $[L, R]$.
\end{itemize}

Тогда из предыдущих фактов следует, что верно

\begin{itemize}
    \item $dp_{L, L} = intersect_{L, L}$
    \item $L \ne R \Rightarrow dp_{L, R} = intersect_{L, R} - 2 \cdot include_{L, R} +
    \min \limits_{L \leqslant m < R} {dp_{L, m} + dp_{m + 1, R}}$
\end{itemize}

$dp_{L, R}$ зависит только от меньших по длине отрезков, поэтому можно найти минимальный вес всего дерева отрезков, считая $dp_{L, R}$ по возрастанию длины отрезков.

Если запоминать для каждого отрезка, какое $m$ дало минимальную сумму значений у сыновей, то можно и восстановить само дерево.

Без подсчёта $intersect_{L, R}$ и $include_{L, R}$ это решение будет работать за $O(n^3)$ времени и $O(n^2)$ памяти.

Самый простой способ подсчёта этих значений - перебор для каждого отрезка всех запросов. Это работает за $O(n^2 S)$, что при $S = O(n^2)$ превращается в $O(n^4)$.

\subsection{Оптимизация до $O(n^3 + S)$ времени}

Для того чтобы получить желаемую асимптотику, мы научимся находить все $intersect_{L, R}$ и $include_{L, R}$ за $O(n^2 + S)$ времени суммарно.

Введем новую величину $exact_{L, R}$ - суммарный вес запросов с границами $[L, R]$. Все её значения очевидно ищутся за $O(n^2 + S)$.

Тогда утверждаются следующие факты про $include_{L, R}$:

\begin{itemize}
    \item $include_{1, n} = exact_{1, n}$
    \item $R \ne n \Rightarrow include_{1, R} = include_{1, R + 1} + exact_{1, R}$
    \item $L \ne 1 \Rightarrow include_{L, n} = include_{L - 1, n} + exact_{L, n}$
    \item $L \ne 1 \land R \ne n \Rightarrow$
\end{itemize}

Первый пункт очевиден. Второй пункт верен, так как любой запрос, содержащий префикс массива, является этим префиксом или большим. Третий пункт верен, так как любой запрос, содержащий суффикс массива, является этим суффиксом или большим.

В последнем пункте верно, что любой запрос, содержащий $[L, R]$, либо совпадает с этим отрезком, либо содержит отрезок с длиной увеличенной на один. При этом два раза считаются отрезки, содержащие $[L - 1, R + 1]$, их надо вычесть.

Таким образом, зная $exact_{L, R}$ можно пересчитывать $include_{L, R}$ по убыванию длины отрезка $[L, R]$, и все значения ищутся за $O(n^2)$ времени.

Теперь утверждаются следующие факты про $intersect_{L, R}$:

\begin{itemize}
    \item $intersect_{L, L} = include_{L, L}$
    \item $L \ne R \Rightarrow intersect_{L, R} = include_{L, L} + intersect_{L + 1, R} - include_{L, L + 1}$
\end{itemize}

Первый факт верен, так как пересечение с отрезком длины один, это то же самое, что и содержание его внутри себя.

Второй факт верен, так как сложив $include_{L, L}$ и $intersect_{L + 1, R}$ мы учтём все пересечения, но отрезки, содержащие отрезок $[L, L + 1]$ посчитается и там, и там.

Зная $include_{L, R}$ можно пересчитывать $intersect_{L, R}$ по возрастанию длины отрезка $[L, R]$, и все значения ищутся за $O(n^2)$ времени.

Таким образом, мы находим все вспомогательные величины для подсчёта $dp_{L, R}$ за $O(n^2 + S)$, что и приводит к решению за $O(n^3 + S)$ времени и $O(n^2)$ памяти.

\subsection{Предпосылки для существования более оптимального решения}

Сейчас наше решение выглядит как подсчёт значений $dp_{L, L}$, где

\begin{itemize}
    \item $dp_{L, L} = cost_{L, L}$
    \item $dp_{L, R} = cost_{L, R} + \min \limits_{L \leqslant m < R} {dp_{L, m} + dp_{m + 1, R}}$
\end{itemize}

Заметим, что суммарная стоимость листьев $cost_{L, L}$ не зависит от формы дерева, поэтому мы можем её поменять произвольным образом, и это не повлияет на оптимальность решения. Поэтому будем считать, что для всех $L, R$ верно $cost_{L, R} = intersect_{L, R} - 2 \cdot include_{L, R}$.

В \cite{yao1980} было показано, что все $dp_{L, R}$ можно найти за $O(n^2)$ времени и памяти, если выполняются два неравенства:

\begin{itemize}
    \item $\forall a \leqslant b \leqslant c \leqslant d:\ cost_{a, d} \geqslant cost_{b, c}$ (монотонность)
    \item $\forall a \leqslant b \leqslant c \leqslant d:\ cost_{a, c} + cost_{b, d} \leqslant cost_{a, d} + cost_{b, c}$ (quadrangle неравенство)
\end{itemize}

Я собираюсь доказать, что в нашем случае первое неравенство верно, а второе верно с противоположным знаком.

Если увеличить отрезок, то количество пересечений с ним не может уменьшиться, а количество запросов, содержащих его, не может увеличиться. Отсюда сразу следует неравенства о монотонности для $intersect_{L, R}$ и $-2 \cdot include_{L, R}$, а значит сложив эти два неравенства, мы получим неравенство о монотонности для $cost_{L, R}$.

Пусть $a \leqslant b \leqslant c \leqslant d$. Рассмотрим как запрос $[L, R]$ может повлиять на левую и правую часть второго неравенства для пересечений, а именно $intersect_{a, c} + intersect_{b, d}$ и $intersect_{a, d} + intersect_{b, c}$.

\begin{itemize}
    \item $R < a \lor L > d \Rightarrow$ не влияет на обе части
    \item $a \leqslant R < b \Rightarrow$ запрос пересекается с $[a, c]$ и $[a, d]$, но не с другими двумя отрезками, поэтому учитывается одинаково в левой и правой части
    \item $c < L \leqslant d \Rightarrow$ запрос пересекается с $[b, d]$ и $[a, d]$, но не с другими двумя отрезками, поэтому учитывается одинаково в левой и правой части
    \item иначе запрос пересекается с $[b, d]$, а значит и со всеми остальными отрезками, и опять же учитывается одинаково.
\end{itemize}

Отсюда верно $intersect_{a, c} + intersect_{b, d} = intersect_{a, d} + intersect_{b, c}$.

Пусть снова $a \leqslant b \leqslant c \leqslant d$. Рассмотрим как запрос $[L, R]$ может повлиять на левую и правую часть второго неравенства для включений, а именно $include_{a, c} + include_{b, d}$ и $include_{a, d} + include_{b, c}$.

\begin{itemize}
    \item Содержит ровно один из отрезков $[a, c]$ и $[b, d] \Rightarrow$
    пересекается с $[b, c]$ и не пересекается с $[a, d] \Rightarrow$ учитывается и слева, и справа по одному разу
    \item Содержит отрезки $[a, c]$ и $[b, d] \Rightarrow$
    пересекается со всеми отрезками и учитывается с обеих сторон одинаково
    \item Не содержит отрезки $[a, c]$ и $[b, d] \Rightarrow$ либо не пересекается ни с одним из отрезков, либо пересекается только с $[b, c]$, что увеличивает только правую часть
\end{itemize}

Отсюда $include_{a, c} + include_{b, d} \leqslant include_{a, d} + include_{b, c}$. Домножив это неравенство на минус два и прибавив равенство для пересечений, получаем неравенство $cost_{a, c} + cost_{b, d} \geqslant cost_{a, d} + cost_{b, c}$

Итак, мы доказали оба неравенства, которые хотели доказать. Я считаю, что они могут оказаться важными для дальнейшего поиска более оптимального точного решения.