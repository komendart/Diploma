\documentclass[14pt]{extarticle}

\usepackage[russian]{babel}
\usepackage[utf8]{inputenc}
\usepackage[T2A]{fontenc}
\usepackage{amsfonts}
\usepackage{amsmath}
\usepackage[
	left 	= 	30	mm,
	right 	=	15	mm,
	top 	=	20	mm,
	bottom 	=	20	mm,
]{geometry}

\setlength{\parindent}{1.25 cm}
\usepackage{indentfirst}

\usepackage[toc,page]{appendix}

\renewcommand{\baselinestretch}{1.5}

\usepackage{titlesec}

\titleformat{\section}
  {\normalfont\fontsize{14}{14}\bfseries}{\thesection}{1em}{}

\titleformat{\subsection}
  {\normalfont\fontsize{14}{14}\bfseries}{\thesubsection}{1em}{}

\usepackage[intoc]{nomencl}
\renewcommand{\nomname}{Обозначения и сокращения}
\makenomenclature
	
%%% Математика

% Шрифты для математики
\usepackage{amsmath}
\usepackage{amsfonts}
\usepackage{amssymb}
\usepackage{cancel}
\usepackage{mathrsfs}
\usepackage{mathtools}
\usepackage{upgreek}
\usepackage{xfrac}


%%% Иллюстрации
\usepackage{graphicx}
\usepackage{subcaption}
\usepackage{wrapfig}
\usepackage[export]{adjustbox}
%\graphicspath{{./img/}}
				
%Подписи
\usepackage		[margin		= 10	pt,
%					font		= footnotesize, 
					labelfont	= bf, 
					labelsep	= endash, 
					labelfont	= bf,
%					textfont	= sl,
					margin		= 0 	pt,  
					aboveskip 	= 4		pt, 
					belowskip 	= -6	pt,
					figurename= Рисунок] {caption}
\usepackage		[margin		= 10	pt,
					font		= footnotesize, 
					labelfont	= bf, 
					labelsep	= endash, 
					labelfont	= bf,
					textfont	= sl,
					margin		= 0 	pt,  
					aboveskip 	= 4		pt, 
					belowskip 	= 6	pt]	{subcaption}

%%% Insert pdf pages
\usepackage[final]{pdfpages}


%%% Color highlight
\usepackage{xcolor}


\begin{document}
%	\includepdf[pages=-]{titlesheet.pdf}
	
	\setcounter{page}{2}
	% Аннотация
	\begin{abstract}
		Дерево отрезков - это мощная структура данных, которая позволяет эффективно решать множество задач различных тематик. Данный диплом посвящен построению оптимального дерева в случае известного распределения вероятностей запросов. Было предложено как точное решение, так и асимптотически оптимальное приближенное решение. На наших тестах приближенное решение показало достаточно небольшое ухудшение по сравнению с оптимальным.
	\end{abstract}
	\newpage
	
	% Содержание
	\tableofcontents
	\newpage
	
	\section{Введение}
	\subsection{Определения}

Пусть дан массив из $n$ элементов.

\textbf{Дерево отрезков} - это двоичное дерево, в котором:

\begin{itemize}
    \item Есть $n$ листьев, соответствующих отрезкам единичной длины
    \item Есть вершины с двумя сыновьями. Правый сын соответствует отрезку,
    следующему сразу за отрезком левого сына. Вершина соответствует
    объединению отрезков сыновей
    \item Корень дерева соответствует всему массиву (отрезку $[1; n]$).
\end{itemize}

\textbf{Запрос сверху} на отрезке $[L, R]$ начинается в корне.
Если сейчас рассматривается вершина, отрезок которой не лежит полностью в
отрезке $[L, R]$, то запрос рекурсивно вызывается от тех сыновей, отрезки которых
пересекаются с $[L, R]$. Иначе рекурсивных вызовов от сыновей не происходит.
В обоих случаях вершина считается посещенной и в ней выполняются какие-то
действия, специфичные для запроса.

Пример запроса сверху указан на рисунке \ref{fig:segtree_example}. Здесь дерево отрезков на пяти элементах и запрос на отрезке $[2;4]$. Будет посещено пять выделенных вершин.

\begin{figure}[hbt!]
    \centering
    \includegraphics[scale=0.28]{images/segtree_example.png}
    \caption{Пример запроса к дереву отрезков}
    \label{fig:segtree_example}
\end{figure}

\subsection{Постановка задачи}

\textbf{Дано:} Распределение вероятностей на запросах-отрезках границами из $[1; n]$

\textbf{Необходимо:} построить дерево отрезков, для которого минимально среднее
количество посещенных вершин при запросах сверху.

Интересует как точное решение за как можно более лучшую асимптотику, так и
приближенное за сложность нахождения $O(n + S)$ или $O(n + S \log S)$, где $S$ -
количество отрезков с ненулевой вероятностью

\subsection{Мотивация}

Дерево отрезков - это мощная структура данных, которая позволяет решать большое количество задач. 
Если дан массив $a$ из $n$ элементов, то она позволяет эффективно:

\begin{itemize}
    \item Отвечать на запросы $f(a_L, f(\dots, a_R)\dots)$, где $f$ - произвольная ассоциативная функция с двумя аргументами
    \item Изменять элементы $a_L,\dots,a_R$ для некоторых типов изменений
\end{itemize}

Пример: запросы суммы/минимума на подотрезке массива и прибавления числа ко всем элементам подотрезка массива.

Обычно для этого в каждой вершине дерева отрезков хранится значение функции на соответствующем подотрезке массива и информация о том, нужно ли применить какие-то отложенные изменения к сыновьям данной вершины.

Запросы на расчёт функции находятся с помощью запроса сверху и объединения результатов для посещенных вершин. Запросы на изменение тоже совершаются с помощью запросов сверху, для посещенных вершин происходит изменение информации в них нужным образом.

Если для каждой вершины дерева с двумя сыновьями её отрезок разбивается примерно пополам и каждой половине соответствуют её сыновья, то можно показать, что при запросе сверху посещается $O(\log n)$ вершин.

При этом большое количество задач различных тематик можно свести к запросам на подотрезках массива. Примеры таких задач:

\begin{itemize}
    \item Для дерева из $n$ вершин после препроцессинга за $O(n)$ можно отвечать на запросы минимального общего предка за $O(\log n)$
    \item Для строки из $n$ символов после построения суффиксного массива и дополнительного препроцессинга за $O(n)$ можно находить длину наибольшего общего префикса двух любых её подстрок
    \item Если нам дано $n$ прямоугольников на плоскости со сторонами, параллельными осям координат, то можно найти площадь их объединения за $O(n \log n)$
\end{itemize}

Как уже было сказано, если для каждой вершины разбивать её подотрезок на две равные части, то сложность запроса к дереву отрезков $O(\log n)$. Но нам может быть известно распределение вероятностей запросов, и в таком случае можно построить более оптимальное дерево. Таким образом решение задачи, поставленной в дипломе, может привести к ускорению на большом классе задач и может представлять не только теоретический, но и практический интерес.

Кроме того надо заметить, что данный диплом далеко не первый, посвященный вопросу построения оптимальных структур данных при известном распределении вероятностей запросов. В частности, смежная задача построения статически оптимального дерева поиска хорошо изучена, её результаты будут использоваться в дальнейшем.
    \newpage
    
    \section{Обзор смежной задачи}
    Это обзор смежной задачи TODO
    \newpage
    
    \section{Точное решение за $O(n^3+S)$ времени и $O(n^2)$ памяти}
    \subsection{Решение за $O(n^3 + n^2S)$ времени}

Пусть вес запроса - это то же самое, что и его вероятность.

Пусть вес дерева отрезков - это сумма по всем данным запросам произведения их веса на количество посещенных вершин в дереве отрезков.

Введём понятие дерева отрезков, построенного на $[L, R]$ и его веса.
Корень такого дерева отрезков соответствует отрезку $[L, R]$. Запросы, не пересекающиеся с $[L, R]$ убираются из рассмотрения, а для остальных берется их пересечение с отрезком $[L, R]$.

Дальше несколько фактов:

\begin{itemize}
    \item Для отрезка $[L, R]$ к весу дерева отрезков прибавляется сумма весов запросов, которые пересекаются с $[L, R]$
    \item Если сыновьям корня соответствуют отрезки $[L, m]$ и $[m + 1, R]$, то в весе дерева отрезков на $[L, R]$ надо
    учесть вес дерева отрезков на $[L, m]$ и на $[m + 1, R]$.
    \item При этом запросы целиком содержащие $[L, R]$ не посещают никаких вершин кроме корня, поэтому их нужно
    вычесть из весов деревьев отрезков на $[L, m]$ и на $[m + 1, R]$.
\end{itemize}
    
Введем несколько обозначений:

\begin{itemize}
    \item $include_{L, R}$ - суммарный вес запросов, которые содержат отрезок $[L, R]$
    \item $intersect_{L, R}$ - суммарный вес запросов, которые пересекаются с отрезком $[L, R]$
    \item $dp_{L, R}$ - минимальный вес дерева отрезков, построенного на $[L, R]$.
\end{itemize}

Тогда из предыдущих фактов следует, что верно

\begin{itemize}
    \item $dp_{L, L} = intersect_{L, L}$
    \item $L \ne R \Rightarrow dp_{L, R} = intersect_{L, R} - 2 \cdot include_{L, R} +
    \min \limits_{L \leqslant m < R} {dp_{L, m} + dp_{m + 1, R}}$
\end{itemize}

$dp_{L, R}$ зависит только от меньших по длине отрезков, поэтому можно найти минимальный вес всего дерева отрезков, считая $dp_{L, R}$ по возрастанию длины отрезков.

Если запоминать для каждого отрезка, какое $m$ дало минимальную сумму значений у сыновей, то можно и восстановить само дерево.

Без подсчёта $intersect_{L, R}$ и $include_{L, R}$ это решение будет работать за $O(n^3)$ времени и $O(n^2)$ памяти.

Самый простой способ подсчёта этих значений - перебор для каждого отрезка всех запросов. Это работает за $O(n^2 S)$, что при $S = O(n^2)$ превращается в $O(n^4)$.

\subsection{Оптимизация до $O(n^3 + S)$ времени}

Для того чтобы получить желаемую асимптотику, мы научимся находить все $intersect_{L, R}$ и $include_{L, R}$ за $O(n^2 + S)$ времени суммарно.

Введем новую величину $exact_{L, R}$ - суммарный вес запросов с границами $[L, R]$. Все её значения очевидно ищутся за $O(n^2 + S)$.

Тогда утверждаются следующие факты про $include_{L, R}$:

\begin{itemize}
    \item $include_{1, n} = exact_{1, n}$
    \item $R \ne n \Rightarrow include_{1, R} = include_{1, R + 1} + exact_{1, R}$
    \item $L \ne 1 \Rightarrow include_{L, n} = include_{L - 1, n} + exact_{L, n}$
    \item $L \ne 1 \land R \ne n \Rightarrow$
\end{itemize}

Первый пункт очевиден. Второй пункт верен, так как любой запрос, содержащий префикс массива, является этим префиксом или большим. Третий пункт верен, так как любой запрос, содержащий суффикс массива, является этим суффиксом или большим.

В последнем пункте верно, что любой запрос, содержащий $[L, R]$, либо совпадает с этим отрезком, либо содержит отрезок с длиной увеличенной на один. При этом два раза считаются отрезки, содержащие $[L - 1, R + 1]$, их надо вычесть.

Таким образом, зная $exact_{L, R}$ можно пересчитывать $include_{L, R}$ по убыванию длины отрезка $[L, R]$, и все значения ищутся за $O(n^2)$ времени.

Теперь утверждаются следующие факты про $intersect_{L, R}$:

\begin{itemize}
    \item $intersect_{L, L} = include_{L, L}$
    \item $L \ne R \Rightarrow intersect_{L, R} = include_{L, L} + intersect_{L + 1, R} - include_{L, L + 1}$
\end{itemize}

Первый факт верен, так как пересечение с отрезком длины один, это то же самое, что и содержание его внутри себя.

Второй факт верен, так как сложив $include_{L, L}$ и $intersect_{L + 1, R}$ мы учтём все пересечения, но отрезки, содержащие отрезок $[L, L + 1]$ посчитается и там, и там.

Зная $include_{L, R}$ можно пересчитывать $intersect_{L, R}$ по возрастанию длины отрезка $[L, R]$, и все значения ищутся за $O(n^2)$ времени.

Таким образом, мы находим все вспомогательные величины для подсчёта $dp_{L, R}$ за $O(n^2 + S)$, что и приводит к решению за $O(n^3 + S)$ времени и $O(n^2)$ памяти.

\subsection{Предпосылки для существования более оптимального решения}

Сейчас наше решение выглядит как подсчёт значений $dp_{L, L}$, где

\begin{itemize}
    \item $dp_{L, L} = cost_{L, L}$
    \item $dp_{L, R} = cost_{L, R} + \min \limits_{L \leqslant m < R} {dp_{L, m} + dp_{m + 1, R}}$
\end{itemize}

Заметим, что суммарная стоимость листьев $cost_{L, L}$ не зависит от формы дерева, поэтому мы можем её поменять произвольным образом, и это не повлияет на оптимальность решения. Поэтому будем считать, что для всех $L, R$ верно $cost_{L, R} = intersect_{L, R} - 2 \cdot include_{L, R}$.

В \cite{yao1980} было показано, что все $dp_{L, R}$ можно найти за $O(n^2)$ времени и памяти, если выполняются два неравенства:

\begin{itemize}
    \item $\forall a \leqslant b \leqslant c \leqslant d:\ cost_{a, d} \geqslant cost_{b, c}$ (монотонность)
    \item $\forall a \leqslant b \leqslant c \leqslant d:\ cost_{a, c} + cost_{b, d} \leqslant cost_{a, d} + cost_{b, c}$ (quadrangle неравенство)
\end{itemize}

Я собираюсь доказать, что в нашем случае первое неравенство верно, а второе верно с противоположным знаком.

Если увеличить отрезок, то количество пересечений с ним не может уменьшиться, а количество запросов, содержащих его, не может увеличиться. Отсюда сразу следует неравенства о монотонности для $intersect_{L, R}$ и $-2 \cdot include_{L, R}$, а значит сложив эти два неравенства, мы получим неравенство о монотонности для $cost_{L, R}$.

Пусть $a \leqslant b \leqslant c \leqslant d$. Рассмотрим как запрос $[L, R]$ может повлиять на левую и правую часть второго неравенства для пересечений, а именно $intersect_{a, c} + intersect_{b, d}$ и $intersect_{a, d} + intersect_{b, c}$.

\begin{itemize}
    \item $R < a \lor L > d \Rightarrow$ не влияет на обе части
    \item $a \leqslant R < b \Rightarrow$ запрос пересекается с $[a, c]$ и $[a, d]$, но не с другими двумя отрезками, поэтому учитывается одинаково в левой и правой части
    \item $c < L \leqslant d \Rightarrow$ запрос пересекается с $[b, d]$ и $[a, d]$, но не с другими двумя отрезками, поэтому учитывается одинаково в левой и правой части
    \item иначе запрос пересекается с $[b, d]$, а значит и со всеми остальными отрезками, и опять же учитывается одинаково.
\end{itemize}

Отсюда верно $intersect_{a, c} + intersect_{b, d} = intersect_{a, d} + intersect_{b, c}$.

Пусть снова $a \leqslant b \leqslant c \leqslant d$. Рассмотрим как запрос $[L, R]$ может повлиять на левую и правую часть второго неравенства для включений, а именно $include_{a, c} + include_{b, d}$ и $include_{a, d} + include_{b, c}$.

\begin{itemize}
    \item Содержит ровно один из отрезков $[a, c]$ и $[b, d] \Rightarrow$
    пересекается с $[b, c]$ и не пересекается с $[a, d] \Rightarrow$ учитывается и слева, и справа по одному разу
    \item Содержит отрезки $[a, c]$ и $[b, d] \Rightarrow$
    пересекается со всеми отрезками и учитывается с обеих сторон одинаково
    \item Не содержит отрезки $[a, c]$ и $[b, d] \Rightarrow$ либо не пересекается ни с одним из отрезков, либо пересекается только с $[b, c]$, что увеличивает только правую часть
\end{itemize}

Отсюда $include_{a, c} + include_{b, d} \leqslant include_{a, d} + include_{b, c}$. Домножив это неравенство на минус два и прибавив равенство для пересечений, получаем неравенство $cost_{a, c} + cost_{b, d} \geqslant cost_{a, d} + cost_{b, c}$

Итак, мы доказали оба неравенства, которые хотели доказать. Я считаю, что они могут оказаться важными для дальнейшего поиска более оптимального точного решения.
    \newpage
    
    \section{Асимптотически оптимальное приближенное решение}
    \subsection{Идея приближения}

\subsection{Доказательство ухудшения не более чем в константу раз}

\subsection{Алгоритм нахождения приближенного решения}

\subsection{Изучение поведения на реальных тестах}
    \newpage

	\section{Заключение}
	Было найдено точное решение задачи за $O(n^3 + S)$ и асимптотически оптимальное приближенное решение. Также были доказаны два неравенства, почти совпадающие с теми, которые используются для доказательства существования решения за $O(n^2)$ для класса задач.

Все найденные решения были написаны и проверены на достаточно небольших случайных тестах, на которых у приближенного решения константа ухудшения ответа по сравнению с оптимальным оказалась достаточно невелика, чтобы данное решение могло иметь не только теоретический, но и практический интерес.

Возможные улучшения - доказательство лучшей константы у приближенного решения и нахождение еще более оптимального точного решения.
	\newpage
	
	% Список литературы
	\begin{thebibliography}{9} 
	\addcontentsline{toc}{section}{Список литературы}
	\bibitem{knuth71} Knuth, Donald E, \emph{"Optimum binary search trees"}. Acta Informatica, 1971.
	\bibitem{mehlhorn1975} Mehlhorn, Kurt, \emph{"Nearly optimal binary search trees"}. Acta Informatica, 1975.
	\bibitem{garsia1977} Garsia, Adriano M, Wachs, Michelle L, \emph{ new algorithm for minimum cost binary trees"}. SIAM Journal on Computing, 1977.
	\bibitem{yao1980} F. Frances Yao, \emph{"Efficient Dynamic Programming Using Quadrangle Inequalities"}. STOC, 1980.
\end{thebibliography}
\end{document}